%!TEX TS-program = xelatex
%!TEX encoding = UTF-8 Unicode
% Awesome CV LaTeX Template for Cover Letter
%
% This template has been downloaded from:
% https://github.com/posquit0/Awesome-CV
%
% Authors:
% Claud D. Park <posquit0.bj@gmail.com>
% Lars Richter <mail@ayeks.de>
%
% Template license:
% CC BY-SA 4.0 (https://creativecommons.org/licenses/by-sa/4.0/)
%


%-------------------------------------------------------------------------------
% CONFIGURATIONS
%-------------------------------------------------------------------------------
% A4 paper size by default, use 'letterpaper' for US letter
\documentclass[11pt, a4paper]{awesome-cv}

% Configure page margins with geometry
\geometry{left=1.6cm, top=1.6cm, right=1.6cm, bottom=1.8cm, footskip=.5cm}

% Color for highlights
% Awesome Colors: awesome-emerald, awesome-skyblue, awesome-red, awesome-pink, awesome-orange
%                 awesome-nephritis, awesome-concrete, awesome-darknight
\definecolor{awesome-purple}{HTML}{b5179e}
\colorlet{awesome}{awesome-purple}
% Uncomment if you would like to specify your own color
% \definecolor{awesome}{HTML}{CA63A8}

% Colors for text
% Uncomment if you would like to specify your own color
% \definecolor{darktext}{HTML}{414141}
% \definecolor{text}{HTML}{333333}
% \definecolor{graytext}{HTML}{5D5D5D}
% \definecolor{lighttext}{HTML}{999999}
% \definecolor{sectiondivider}{HTML}{5D5D5D}

% Set false if you don't want to highlight section with awesome color
\setbool{acvSectionColorHighlight}{true}

% If you would like to change the social information separator from a pipe (|) to something else
\renewcommand{\acvHeaderSocialSep}{\quad\textbar\quad}


%-------------------------------------------------------------------------------
%	PERSONAL INFORMATION
%	Comment any of the lines below if they are not required
%-------------------------------------------------------------------------------
% Available options: circle|rectangle,edge/noedge,left/right
% \photo[circle,noedge,left]{./examples/profile}
\name{Rishi}{Raj}
\position{Gradudate Researcher{\enskip\cdotp\enskip}Theoretical Physics}
\address{Department of Physics, IIT Madras, Chennai, India}

\mobile{(+91) 79829-36310}
\email{rishiraj.1012exp@gmail.com}
%\dateofbirth{January 1st, 1970}
% \homepage{www.posquit0.com}
\github{rshrj}
\linkedin{rshrjnc}
% \gitlab{gitlab-id}
% \stackoverflow{SO-id}{SO-name}
% \twitter{@twit}
% \skype{skype-id}
% \reddit{reddit-id}
% \medium{madium-id}
% \kaggle{kaggle-id}
% \googlescholar{googlescholar-id}{name-to-display}
%% \firstname and \lastname will be used
% \googlescholar{googlescholar-id}{}
% \extrainfo{extra information}

% \quote{``Be the change that you want to see in the world."}


%-------------------------------------------------------------------------------
%	LETTER INFORMATION
%	All of the below lines must be filled out
%-------------------------------------------------------------------------------
% The company being applied to
\recipient
  {Graduate Recruitment Team}
  {Uppsala universitet\\752 36 Uppsala, Sweden}
% The date on the letter, default is the date of compilation
\letterdate{\today}
% The title of the letter
\lettertitle{Application for the PhD position in Theoretical Physics}
% How the letter is opened
\letteropening{Dear Dr. Longhi,}
% How the letter is closed
\letterclosing{Sincerely,}
% Any enclosures with the letter
\letterenclosure[Attached]{Curriculum Vitae}


%-------------------------------------------------------------------------------
\begin{document}

% Print the header with above personal information
% Give optional argument to change alignment(C: center, L: left, R: right)
\makecvheader[R]

% Print the footer with 3 arguments(<left>, <center>, <right>)
% Leave any of these blank if they are not needed
\makecvfooter
  {\today}
  {Rishi Raj~~~·~~~Cover Letter}
  {}

% Print the title with above letter information
\makelettertitle

%-------------------------------------------------------------------------------
%	LETTER CONTENT
%-------------------------------------------------------------------------------
\begin{cvletter}

  As a child, I had two conflicting formative environments where my family followed traditional Indian customs and superstitions that defied logic, and the school environment taught me to challenge my preconceived notions. I struggled against irrational thought processes rooted in ancient beliefs and found solace watching Feynman's and V Balakrishnan's lectures during high school. Starting from basic Newton's laws and some physical insight, I could understand heat engines, gases, solids, and all seemingly complicated objects. I found patterns in nature and abstract intriguing and jumped at opportunities to dissect and understand these structures. 

  In the last few years, I have become more inclined towards supersymmetric field theories and related mathematics of algebraic geometry. By producing substantial graduate research work, I want to push the frontiers at the cutting edge of quantum field theories and supersymmetry. Being mentored by Prof Pietro would allow me to learn and practice deep aspects of supersymmetric field theories and strengthen my work. 
  
  The close collaboration with the Geometry and Physics group at Uppsala puts me in an advantageous position to have a scholarly exchange that could help me with my work. My work would involve algebraic geometry and homological mirror symmetry. Attending the talks and discussions with brilliant people at Uppsala will benefit me. I will also have access to Uppsala's HPC to do performance-intensive computations. On top of that, Uppsala has excellent cold weather with comfortable summers, and the university is a beautiful place to spend the next 4-5 years of my life. 
  
  In my first year, I was regularly challenged by my supervisor Ayan Mukhopadhyay with simple but tricky problems related to his research. In one instance, I was asked to understand the so-called semi-holographic approach that he and collaborators developed that involves coupling systems through their effective spacetime metrics. Through my beginner-level understanding of Differential Geometry in my first year, I could derive the coupling equations for a simple toy example of a scalar field coupled to a perfect fluid. I then used my computation skills in Mathematica to simulate and study the system's response under perturbations. This was the first research experience that I very much enjoyed.
  
  I spent the summer of my second year at a national science camp called NIUS at HBCSE, TIFR, Mumbai. I attended various talks and workshops on cutting-edge physics and astronomy and conducted lab experiments. I was then assigned a research project for the winter, which was more on the experimental side: to assemble and study a system of magnetically coupled pendula. This was partly meant to be an educational demonstration. I had little inclination or experience in setting up experiments, so this was the perfect opportunity to learn something outside my comfort zone. I used relatively easy-to-find parts to assemble both the system and a reasonably precise measuring apparatus that involved, among other things, motion tracking the position of the pendula. After carefully taking many samples, I turned to the expectations: I set up the equations of motion: linearizing them near the stable equilibria to find the normal modes, which was done both approximately and exactly using Mathematica. In the end, I found a more than 99% accurate fit.
  
  I then worked on more serious research projects with Ayan. In the Fall semester of my third year, I started the quest to understand the features of black holes through quantum mechanical matrix models. This is the most ambitious project I have undertaken so far. I worked through various seminal papers to understand known aspects of these models, such as how they arise through the second quantization of supermembranes and how one-loop quantum effects famously lead to the classical supergravity interactions. I then made many attempts at numerically probing the classical dynamics of these matrices for various matrix sizes and a number of matrices. My goal was to do this at a reasonably large matrix size since this is the limit where we have most of our physical insights, but this is highly non-trivial since the number of variables to solve for increases quadratically with the matrix size. Even worse, the classical equations scale with the sixth power. This quickly becomes completely intractable in Mathematica. To get around this, I implemented the numerical procedure using low-level C/C++ code to optimize as much as possible. I was able to achieve a 17x boost in simulation times, on top of being able to utilize parallel computation and HPC clusters. We now have a fairly varied dataset of classical simulation data for large matrix sizes and are now working on adding quantum corrections and looking at things like route to thermalization.
  
  In the summer of my fourth year, I did a short but intense research internship at the Laboratory of Theoretical and High Energy Physics (LPTHE) in Jussieu, Paris, under the supervision of Boris Pioline. I worked on Wall Crossing and related things in N=2 Supergravity in four dimensions. This project was closest to my personal taste because it involved cutting-edge geometry and mathematics applied to theoretical physics and vice versa. It also brought me closer to the microstate interpretation of Hawking-Bekenstein black hole entropy, a fundamental open problem in physics. I used my basic familiarity with algebraic geometry and supersymmetry to understand the setup fairly well. During the internship, I implemented a host of discrete algorithms that computed the refined BPS index in two different ways. From these little experiments, I was able to point out an interesting connection: a partitioning of the phase space of multi-centered black hole solutions in supergravity (a symplectic manifold) coming from the corresponding attractor flow trees (a discrete structure used in computing the index). We then studied certain cases where this manifold develops a conical singularity, and the two different methods of computing the index yielded different answers.
  
  My undergraduate preparation has trained me to be a highly productive and ambitious researcher. The variety of research projects I have undertaken, along with the rigorous coursework and self-study, has given me a solid foundation in modern theoretical physics: from field theory to string theory and supersymmetry to aspects of theoretical condensed matter physics. Further, my background in mathematics puts me in the right place to tackle the subject's purely mathematical or abstract aspects. Lastly, the computation skills I learned during my projects helped me think highly structurally about the problems, which lends itself to immediate computations and immediate insights into an otherwise seemingly intractable problem.
  
  I have acquired a highly structured thought process that is rigorous enough to keep me on the right track while not being too rigid that I lose my creativity. My style of attacking problems is heavily motivated by Feynman: I almost never do long blind computations with no expectations to compare to, and so whenever possible, I always make sure I have enough insight into the problem to have educated expectations for what I might find.
  
  I  look forward to producing outstanding scientific research in supersymmetry by pursuing my education at Uppsala and thus furthering my commitment to the ideals of rationality and science, which has given my life its greatest purpose.
  
  

\end{cvletter}


%-------------------------------------------------------------------------------
% Print the signature and enclosures with above letter information
\makeletterclosing

\end{document}
