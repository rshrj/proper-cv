%-------------------------------------------------------------------------------
%	SECTION TITLE
%-------------------------------------------------------------------------------
\cvsection{Research Experience}


%-------------------------------------------------------------------------------
%	CONTENT
%-------------------------------------------------------------------------------

\cvsubsection{High Energy Theory}

\begin{cventries}

%---------------------------------------------------------
  \cventry
    {String Theory Group, IITM\footnotemark[1] and Vishnu Jejjala, U. Witwatersrand} % Job title
    {Deep Learning Gravity} % Organization
    {Chennai, India} % Location
    {Oct 2022 - Present} % Date(s)
    {
      \begin{cvitems} % Description(s) of tasks/responsibilities
        \item {Developing high accuracy neural networks to learn basic features of non-equilibrium processes on asymptotically AdS geometries such as the areas and positions of the event and apparent horizons starting from parameters describing numerical solutions of Einstein's equations.}
      \end{cvitems}
    }
    \footnotetext[1]{\footnotestyle{Ayan Mukhopadhyay, Tanay Kibe, Sukrut Mondkar}}


%---------------------------------------------------------
  \cventry
    {String Theory Group, IITM and Vishnu Jejjala, U. Witwatersrand} % Location
    {Black Hole microstates in Matrix models} % Organization
    {Chennai, India} % Date(s)
    {Aug 2020 - Present} % Date(s)
    {
      \begin{cvitems} % Description(s) of tasks/responsibilities
        \item {Developing efficient numerical techniques to perform classical and quantum mechanical simulations of the M-theory matrix model for relatively large matrix sizes.}
        \item {Understanding thermalization features and entanglement using the massive simulation data.}
      \end{cvitems}
    }
    
%---------------------------------------------------------
  \cventry
    {Internship supervised by Boris Pioline at LPTHE, Sorbonne University} % Job title
    {Wall Crossing Phenomena in $\mathcal{N}=2$ SUGRA} % Organization
    {Jussieu, Paris} % Location
    {Summer 2022} % Date(s)
    {
      \begin{cvitems} % Description(s) of tasks/responsibilities
        \item {Implemented various algorithms to compute BPS indices and jumps thereof in the complexified Kähler moduli space of Calabi-Yau threefolds in $\mathcal{N} = 2$ supergravity in four dimensions.}
        \item {Two approaches for computing the indices were shown to be equivalent in certain cases, and discrepancies were identified in a particular class of charges.}
        \item {Identified an interesting connection between the phase space of multi-centered solutions and corresponding attractor flow trees.}
      \end{cvitems}
    }


%---------------------------------------------------------

\cventry
    {Supervised by Ayan Mukhopadhyay, IIT Madras} % Job title
    {Semiholographic Networks} % Organization
    {Chennai, India} % Location
    {Jul 2019 - Nov 2020} % Date(s)
    {
      \begin{cvitems} % Description(s) of tasks/responsibilities
        \item {Developed a simple set of networks of scalar fields coupled with perfect fluids using the Semiholographic approach developed by the supervisor and colleagues.}
        \item {Analyzed the response of the networks to perturbations and quenches.}
      \end{cvitems}
    }


%---------------------------------------------------------
  \cventry
    {Reading projects supervised by Ayan Mukhopadhyay, IIT Madras} % Job title
    {Independent Research} % Organization
    {Chennai, India} % Location
    {Summer 2019} % Date(s)
    {
      \begin{cvitems} % Description(s) of tasks/responsibilities
        \item {Learnt about the Montonen-Olive and similar duality conjectures, the Witten effect in the context of $\mathcal{N}=4$ SYM.}
        \item {Learnt about modern relativistic hydrodynamics and how transport coefficients are significantly constrained by consistency requirements with thermal partition functions in QFTs in stationary background spacetimes.}
        \item {Developed a good understanding of memory effect, soft theorems, and asymptotic symmetries and their relationship in the infrared physics of quantum field theories.}
      \end{cvitems}
    }


%---------------------------------------------------------
\end{cventries}

\cvsubsection{Other}

\begin{cventries}

  \cventry
    {NIUS Internship mentored by Praveen Pathak, HBCSE} % Job title
    {Magnetically Coupled Pendula} % Organization
    {Mumbai, India} % Location
    {Nov 2019 - Jan 2020} % Date(s)
    {
      \begin{cvitems} % Description(s) of tasks/responsibilities
        \item {Created an experimental setup of pendula with cylindrical bar magnets attached to their ends and confined to move in a plane.}
        \item {Wrote down a simple theoretical model for the system and compared it with the data taken experimentally, systematically taking care of biases and errors. Found a reasonable and expected degree of experimental agreement.}
      \end{cvitems}
    }


%---------------------------------------------------------

\end{cventries}
